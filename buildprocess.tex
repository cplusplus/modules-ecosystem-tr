%!TEX root = iso_cpp_modules_ecosystem_technical_report.tex

\rSec0[build]{Build Process}

\indextext{build|(}%

\rSec1[build.import]{Import relationship between translation units}

\pnum The name of a module used in the ``import'' and ``module''
keywords is the ``logical name'' of a module.

\pnum When a translation unit imports a named module, the translation
unit doing the import has an import dependency on the translation unit
providing the importable unit identified via the ``logical name''.

\pnum The translation of an importable unit is externalized into an
intermediate artifact, the Built Module Interface file (BMI). That
file is then consumed when importing a named module.

\rSec1[build.arguments]{Compiler arguments and the import relationship}

\pnum ``ABI coherency'' is the expected level of coherency in the
compilation command for the translation of the code in the purview of a
module and all translation units importing that module. When that
level of coherency is not met, the final program may fail to link or
it may link and have undefined behavior at runtime.

\pnum ``BMI coherency'' is the expected level of coherency in the
compiler arguments for a BMI file to be used when importing a
module. When that level of coherency is not met, the translation will
fail because the BMI cannot be loaded.

\pnum While all translation units importing a module must have ``ABI
coherency'', different translation units importing the same module may
not meet the requirements for ``BMI coherency''. It must be valid for
those to consume different BMI files of the same importable unit, and
to be linked into the same program.

\pnum ``Local Preprocessor Arguments'' are the arguments that are
expected to be different in the translation of an importable unit and
the translation of a unit importing that named module.

\pnum The decision on whether an argument is a ``Local Preprocessor
Argument'' cannot be made by introspection on the code or the command
line, and should be be provided by the author of the importable units
and of the unit importing a module.

\pnum To assemble a ``BMI coherent'' command line in order to produce
a BMI file that is usable in the context of a translation unit
importing that module, the starting point should be the compilation
command of the translation unit doing the import, then the ``Local
Preprocessor Arguments'' for the translation unit doing the import are
removed, and the ones for the importable unit are integrated into the
command line.

\pnum Compilers should offer an interface to generate an identifier
for the command line that allows matching the translation unit
importing a module with a BMI that can be loaded in that context. That
interface should not require the primary source file of the
translation unit to be identified, nor the output names to be
produced. Any preprocessor argument given to the compiler in this mode
should be considered for the identifier.

\pnum Build systems should be able to deduplicate the rules to
generate BMI files to be used when importing the same named module
from different translation units by getting the identifier for the
command line of those translation units with the ``Local Preprocessor
Arguments'' filtered out.

\rSec1[build.steps]{Steps of the Build}

\pnum When building \Cpp{} code that produces or consumes modules, the
build process needs to be broken down in different steps. The steps
described here represent the semantic organization, not the interface
presented to the \Cpp{} developer, the organization in steps does not
preclude parts of those steps to be executed in parallel.

\rSec2[build.steps.external-importable]{Identify External Importable Units}

\pnum As the \Cpp{} standard library itself will offer a modular
interface as well as importable headers, it will be very rarely the
case that a build system will be able to consider only modules
declared inside the same build context.

\pnum External importable units will be described in metadata files
made available to the build system. While this report recommends a
specific format for that metadata file, the mechanism by which the set
of metadata files to be considered is identified will be
implementation specific.

\pnum Annex \ref{discovery-prebuilt.linker-arguments} specifies a
convention for environments where the build system and the package
manager can resolve linker argument fragments early enough such that
it can be used to identify the set of module metadata files that need
to be considered.

\pnum TODO: Format of the metadata file.

\pnum TODO: \href{https://github.com/cplusplus/modules-ecosystem-tr/issues/5}{modules-ecosystem-tr\#5}

\rSec2[build.steps.importable-headers]{Identify Importable Headers}

\pnum While external importable units may include header units, it's
also necessary to identify any importable header internal to the
project itself.

\pnum The \Cpp{} standard describes ``importable headers'' and
``non-importable headers'', however those cannot be differentiated
from their content alone, therefore they need to identified at the
tooling level.

\pnum The \Cpp{} standard also allows (15.3.7) the implementation to
optimize away source file inclusion when a header is known to be
importable.

\pnum Since an include directive may be replaced by an import
directive, a change in the identification of header units may change
the dependency information of any translation unit.

\pnum TODO: \href{https://github.com/cplusplus/modules-ecosystem-tr/issues/6}{modules-ecosystem-tr\#6}

\rSec2[build.steps.dependency-scan]{Dependency Scanning}

\pnum Once the list of importable headers is defined, the dependency
scanning phase can be performed. The dependency scanning is invoked by
the build system on all translation units (including header units).

\pnum The dependency scanning of each translation unit is independent
of the dependency scanning of any other translation unit.

\pnum TODO: \href{https://github.com/cplusplus/modules-ecosystem-tr/issues/7}{modules-ecosystem-tr\#7}

\rSec2[build.steps.bmi-generation]{Generation of the Binary Module Inteface}

\pnum TODO: \href{https://github.com/cplusplus/modules-ecosystem-tr/issues/8}{modules-ecosystem-tr\#8}

\rSec2[build.steps.compilation]{Compilation}

\pnum TODO: \href{https://github.com/cplusplus/modules-ecosystem-tr/issues/9}{modules-ecosystem-tr\#9}

\rSec2[build.steps.linking]{Linking}

\pnum TODO: \href{https://github.com/cplusplus/modules-ecosystem-tr/issues/10}{modules-ecosystem-tr\#10}

\rSec1[build.coherency]{Coherency Requirements}

\pnum TODO: \href{https://github.com/cplusplus/modules-ecosystem-tr/issues/11}{modules-ecosystem-tr\#11}

\rSec1[build.header-units]{Header Units}

\pnum TODO: \href{https://github.com/cplusplus/modules-ecosystem-tr/issues/12}{modules-ecosystem-tr\#12}

\rSec1[build.stdlib]{Relationship with the Standard Library}

\pnum TODO: \href{https://github.com/cplusplus/modules-ecosystem-tr/issues/13}{modules-ecosystem-tr\#13}

\rSec1[build.trivial]{Trivial Cases}

\pnum TODO: \href{https://github.com/cplusplus/modules-ecosystem-tr/issues/14}{modules-ecosystem-tr\#14}

\indextext{build|)}%
