%!TEX root = iso_cpp_modules_ecosystem_technical_report.tex

\rSec0[introduction]{Introduction}

\rSec1[introduction.scope]{Scope}

\pnum
\indextext{scope|(}%
This aim of this technical report is to describe a model and best practices for
how the C++ software development ecosystem should adopt modules.
\indextext{scope|)}%

\indextext{normative references|see{references, normative}}%
\rSec1[introduction.references]{Normative references}

\pnum
\indextext{references!normative|(}%
The following referenced documents are indispensable for the application of this
document.
For dated references, only the edition cited applies.
For undated references, the latest edition of the referenced document
(including any amendments) applies.

\begin{itemize}
\item ISO/IEC 14882:2020, \doccite{Programming languages --- C++}
\end{itemize}

\pnum
ISO/IEC 14882:2020 is hereafter called the \defn{\Cpp{} Standard}.
\indextext{references!normative|)}

\rSec1[introduction.structure]{Structure of this document}

\indextext{technical report!structure of|(}%

\pnum \ref{source} discusses how tooling relates for the different
types of \Cpp{} source files, the terminology used by tools when
referring to those and how they relate to each other.

\pnum \ref{build process} discusses requirements for tooling related to the
build process of code declaring and/or consuming \Cpp{} modules.

\pnum \ref{language semantics} discusses the interpretation of specific
features described in the \Cpp{} Standard in relationship to modules.

\pnum \ref{distribution} discusses interoperability for the distribution of
\Cpp{} libraries with modules across different compilers, build
systems and package managers.

\indextext{technical report!structure of|)}%
