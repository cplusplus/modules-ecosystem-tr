%!TEX root = iso_cpp_modules_ecosystem_technical_report.tex

\rSec0[introduction]{Introduction}

\rSec1[introduction.scope]{Scope}

\pnum
\indextext{scope|(}%
This aim of this technical report is to describe a model and best practices for
how the C++ software development ecosystem should adopt modules.
\indextext{scope|)}%

\indextext{normative references|see{references, normative}}%
\rSec1[introduction.references]{Normative references}

\pnum
\indextext{references!normative|(}%
The following referenced documents are indispensable for the application of this
document.
For dated references, only the edition cited applies.
For undated references, the latest edition of the referenced document
(including any amendments) applies.

\begin{itemize}
\item ISO/IEC 14882:2020, \doccite{Programming languages --- C++}
\end{itemize}

\pnum
ISO/IEC 14882:2020 is hereafter called the \defn{\Cpp{} Standard}.
\indextext{references!normative|)}

\rSec1[introduction.structure]{Structure of this document}

\pnum
\indextext{technical report!structure of|(}%
\ref{definitions} introduces terms and definitions used throughout this
technical report.

\ref{usage} describes the stakeholders, requirements, and expected usage
patterns for modules that are addressed by this technical report.

\pnum
Each of the remaining clauses addresses a particular topic area.
These clauses present both \defn{findings}, where key questions are
explored and the results of of field experience are presented, and
\defn{guidance}, where concrete suggestions to stakeholders are made.
\indextext{technical report!structure of|)}

