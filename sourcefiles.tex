%!TEX root = iso_cpp_modules_ecosystem_technical_report.tex

\rSec0[source]{Source Files}

\indextext{source files|(}%

\pnum For tooling purposes, there are two main categories of source
files: the main file, and included files. The main file of a
translation unit is the one used to start the first phase of the
translation.  The included files contribute to translation unit by
methods of ``Source file inclusion''.

\pnum With the exception of header units, the translation unit type
can be defined unambiguously by the contents of the main file because
of preprocessing restrictions established by the \Cpp{}
Standard. Header units require additional information at the tooling
level.

\rSec1[source.types]{Translation Unit Types}

\indexdefn{importable units}%
\definition{importable units}{source.types.importable}%
%
Translation units with declarations and definitions that can be
imported into another translation unit via the ``import'' keyword.

\indexdefn{module interface units}%
\definition{module interface units}{source.types.importable.interface}%
%
Importable translation units that contribute to the ``external
interface'' of a named module.

\indexdefn{primary module interface unit}%
\definition{primary module interface unit}{source.types.importable.interface.primary}%
%
The interface unit without a partition, it can be unambiguosly
identified by the presence of the export-keyword and the absense of a
module-partition. There can be only one primary unit for each module.

\indexdefn{module interface partition unit}%
\definition{module interface partition unit}{source.types.importable.interface.partition}%
%
The interface unit with a module-partiton. It can be unambiguously
identified by the presence of both export-keyword and the presence of
module-partition. There can only be one unit for a given partition
name.

\indexdefn{internal module partition unit}%
\definition{internal module partition unit}{source.types.importable.internal.partition}%
%
Importable translation units that do not contribute to the ``external
interface'' of a module. It can be unambiguously identified by the
absense of the export-keyword and the presence of module-partition. An
internal partition may or may not be reachable by the module interface
units. An internal partition that is not reachable by the interface
units is not required to be made available to translation units
outside the module.

\indexdefn{header unit}%
\definition{header unit}{source.types.importable.header}%
%
Header units are the independent parsing of what would otherwise be
available via source inclusion. They are not identifiable directly by
the contents of the main source file. Additional information is needed
at the tooling level to identify them. Implementations are not
required to automatically make the header unit importation possible
for any file that could be otherwise included.

\indexdefn{module implementation unit}%
\definition{module implementation unit}{source.types.implementation}%
%
Module implementation units contain the definitions that were omitted
on the interface units and internal partitions units. It can be
unambiguously identified by the absense of the export-keyword and the
absence of module-partition. It may also contain other declarations,
but those are not reachable from the module interface or internal
partitions.

\indexdefn{non-module unit}%
\definition{non-module unit}{source.types.non-module}%
%
Translation units without a module-declaration that are not header
units.

\indextext{source files|)}%

